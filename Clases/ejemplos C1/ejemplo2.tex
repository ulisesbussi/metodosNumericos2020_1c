\documentclass[10pt]{article}

\usepackage[latin1]{inputenc}
%\usepackage[spanish,activeacute]{babel}
\usepackage[compress]{cite}
\usepackage{graphicx}
\usepackage[usenames,dvipsnames]{color}
\usepackage{amsmath}
\DeclareMathOperator{\sen}{sen}
\usepackage{balance}
\newcommand{\fgref}[1]{Fig.~\ref{#1}}
% Para que el t�tulo referencias aparezca en espa�ol puede usar la siguiente sentencia
 \renewcommand{\refname}{Referencias}
\usepackage[makeroom]{cancel}
% correct bad hyphenation here
\hyphenation{lo-ca-li-za-cio-nes ve-hi-cu-lar cons-truc-cion geo-re-fe-ren-cia-das mu-ni-ci-pal si-mu-la-cio-nes }


\begin{document}

% paper title
% can use linebreaks \\ within to get better formatting as desired
\title{Ejemplos Errores}
\author{Ulises Bussi}


\maketitle

\section{Ejemplo: Tiro oblicuo}

Supongamos que se tira un proyectil con un angulo de {$45^\circ$} con una velocidad inicial de $50m/s$. Calcular la distancia m�xima recorrida.
\begin{itemize}
\item Utilizando todas las cifras significativas.
\item Redondeando a la primer cifra y usando $g=10m/s^2$
\end{itemize}
  
Lo primero que debemos hacer es calcular el tiempo de vuelo. Si consideramos despreciable los rozamientos con el aire, el tiempo de vuelo vendr� dado por la condici�n de que la altura del proyectil vuelva a ser 0. 
La velocidad vertical se puede calcular como $ v_{y0} = v_0 \cos(45^\circ}$



\end{document}


