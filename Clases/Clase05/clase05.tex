%----------------------------------------------------------------------
\documentclass[xcolor=svgnames]{beamer} %, handout
\usetheme{lined}
\usepackage{array}
\usepackage{amsmath}
\usepackage{amssymb}
\usepackage{amsthm}
\usepackage{bbm}
\usepackage[utf8]{inputenc}
\usepackage[T1]{fontenc}
\usepackage{cmbright}   
\usepackage{times}
\usepackage{geometry}
\usepackage[spanish]{babel}
\usepackage{multicol}
\usepackage{tcolorbox}

\usepackage{algorithm2e}

%\usepackage{psfrag}
\usepackage{graphicx}
\usepackage{color}
\usepackage{floatflt}
\usepackage{fancybox}
\usepackage{tabularx}
%\usepackage[all]{xy}
\usepackage{color}
\usepackage{siunitx} % para \degree
\usepackage{mathtools}

\usepackage{tikz}
\usepackage{tikz-cd}
\usetikzlibrary{arrows,matrix,positioning}


\usetikzlibrary{positioning}
\usepackage{arydshln} %poder pone lineas punteadas en matrices




%\usepackage[most]{tcolorbox}


\theoremstyle{plain}
\newtheorem{definicion}{Definición}



\definecolor{redUnq}{rgb}{.7,.1,.1}
\definecolor{redUnq2}{rgb}{.5,.1,.3}

\mode<presentation>{
	%\usetheme{Boxes}
	%\usecolortheme[RGB={237,132,8}]{structure}
	%\usecolortheme[RGB={205,173,0}]{structure}
	\usecolortheme[RGB={100,10,10}]{structure}

	%\beamertemplateshadingbackground{SteelBlue!70}{Honeydew!10}
	%\usetheme{Warsaw}
	%\usecolortheme{default}
	\usetheme{Singapore}
	%\usetheme{Lined}
	%\usetheme[height=7mm]{Rochester} 
	\setbeamerfont{title}{shape=\bfseries,family=\rmfamily}
	%\usefonttheme[onlylarge]{structuresmallcapsserif}
	%\usefonttheme[onlysmall]{structurebold}
	\setbeamercolor{title}{fg=redUnq,bg=gray!40}
	\usefonttheme{professionalfonts}
	\setbeamercovered{highly dynamic}
	\setbeamercovered{transparent=10}
	\setbeamertemplate{navigation symbols}{}
	\colorlet{structure}{redUnq}

	\setbeamertemplate{frametitle}[default][left]
}

\definecolor{verzul}{rgb}{0, 0.5,0.5}

\renewcommand{\textbf}[1]{{\bfseries\textcolor{redUnq2}{#1}}}
\renewcommand{\emph}[1]{{\em\textcolor{redUnq2}{#1}}}

\setlength{\parindent}{0pt}
\theoremstyle{definition}
\newtheorem{ejem}{Ejemplo}
\newtheorem{defi}{Definición}
\newtheorem{ejer}{Ejercicio}
\newtheorem{prop}{Propiedad}
\newtheorem{lema}{Lema}
\newtheorem{teor}{Teorema}
\newtheorem{coro}{Corolario}

 

\newcommand{\Rset}{\mathbbmss{R}}
\newcommand{\Cset}{\mathbbmss{C}}
\newcommand{\PD}[2]{\frac{\partial #1}{\partial #2}}
\DeclareMathOperator{\tr}{tr}
\DeclareMathOperator{\adj}{adj}
\DeclareMathOperator{\rango}{rango}

\newenvironment{Boxedminipage}%
{\begin{Sbox}\begin{minipage}}%
{\end{minipage}\end{Sbox}\fbox{\TheSbox}}



\title{Métodos Numéricos - Clase 3}
  \logo{\includegraphics[scale=0.25]{logoUnq} }
\author{Ulises Bussi- Javier Portillo}
%\institute{\scalebox{2}{\includegraphics[scale=0.1]{mdp02.jpg}}} %{Departamento de Ciencia y Tecnología\\ Universidad Nacional de Quilmes\\ }
\date{ $1^\circ$ cuatrimestre 2020} 


%%%%%%%%% Para que al comenzar una section aparezca el Contenido
%\AtBeginSection[]
%{
%  \begin{frame}
%    \frametitle{Contenidos de la Presentación}
%    \tableofcontents[currentsection]
%  \end{frame}
%}

\AtBeginSection[]
{
    \begin{frame}
        \frametitle{Métodos Iterativos}
        \tableofcontents[currentsection]
    \end{frame}
}


\begin{document} 


\begin{frame} %\thispagestyle{empty}
	\titlepage
\end{frame}

\begin{frame}
\tableofcontents
\end{frame}


\section{Introducción Metodos iterativos}

\begin{frame}
\frametitle{Sistemas de ecuaciones Lineales (SELs).}

\vspace{10pt}


\begin{tcolorbox}
\textbf{Resolver problemas de la forma}
$$A x = b $$
con $ A \in \mathbb{R}^{n\times n}$, $b$ y $x\in \mathbb{R}^{n\times 1}$
\end{tcolorbox} \vspace{20pt}

\textbf{¿Por qué?}
\pause
La mayoría de los problemas de ingeniería pueden llevarse a esta forma.


\end{frame}




\begin{frame}
\frametitle{Métodos iterativos}
\begin{itemize}
\item Aproximar la solución iterando.
\pause
\item Útil para sistemas grandes.

\end{itemize}
\end{frame}



\section{Método de Gauss-Seidel}

\begin{frame}
\frametitle{Método de Gauss-Seidel}

Transformar el sistema:
{\small $$ \begin{bmatrix}
a_{1,1} & a_{1,2} 	& \dots 	 	& a_{1,n} \\
a_{2,1} & a_{2,2} 	& \dots	 	& a_{2,n} \\
\vdots 	& 		  	& \ddots	 	& \vdots \\
a_{n,1}	& a_{n,1} 	& \dots 		& a_{n,n}
\end{bmatrix} \begin{bmatrix}
x_1\\
x_2\\
\vdots\\
x_n
\end{bmatrix} =\begin{bmatrix}
b_1\\
b_2\\
\vdots\\
b_n
\end{bmatrix} \rightarrow \displaystyle \begin{array}{ccc}
x_{\color{red}1} & =& \frac{1}{a_{{\color{red}1,1}}}\left(b_{\color{red}1} - \sum_{\substack{i=1 \\ i\neq{\color{red}1}}}^n  a_{ {\color{red}1},i} x_i \right)\\
x_{\color{red}2} & =& \frac{1}{a_{{\color{red}2,2}}}\left(b_{\color{red}2} - \sum_{\substack{i=1 \\ i\neq{\color{red}2}}}^n  a_{ {\color{red}2},i} x_i\right)\\
 & \vdots & \\
x_{\color{red}n} & =& \frac{1}{a_{{\color{red}n,n}}}\left(b_{\color{red}n} - \sum_{\substack{i=1 \\ i\neq{\color{red}n}}}^n  a_{ {\color{red}n},i} x_i\right)\\
 
\end{array} $$ }

\end{frame}



\begin{frame}
\frametitle{Método de Gauss-Seidel}
Con el sistema de la forma: $\begin{array}{ccc}
x_1 &=& f_1(x_1,x_2,\dots,x_n)\\
x_2 &=& f_2(x_1,x_2,\dots,x_n)\\
&\vdots&\\
x_n &=& f_n(x_1,x_2,\dots,x_n)
\end{array}$
\begin{enumerate} 
\item Proponemos una condición inicial para todos los estados $X^0=[x_1^0,x_2^0,...,x_n^0]$
\item encontramos $x_1^1 = f_1(x_1^0,x_2^0,\dots,x_n^0)$,
\item encontramos $x_2^1 = f_2({x_1}^{\color{red}1},x_2^0,\dots,x_n^0)$, y así sucesivamente.
\end{enumerate}
\pause
Repetimos los pasos [2] y [3]. 

\end{frame}

\begin{frame}
\frametitle{Método de Gauss-Seidel}
Podemos escribir el error para cada una de las incógnitas $x_i$ en el paso $j$ cómo:
$$ e_i^j = \frac{|x_i^j -x_i^{j-1}|}{x_i^j}$$
\end{frame}


\subsection{Ejemplo Gauss-Seidel}

%paso 1
\begin{frame}
\frametitle{Ejemplo: Gauss-Seidel}

Supongamos que tenemos el problema:
$$  \begin{bmatrix}
8 & 3 & 1 \\
1 & 5 & 3 \\
3 & 1 & 5
\end{bmatrix} X = \begin{bmatrix}
12\\
9\\
9
\end{bmatrix} \rightarrow \begin{array}{ccc}
x_1 &=& \frac{12-3x_2-1x_3}{8}\\
x_2 &=& \frac{9-1x_1-3x_3}{5}\\
x_3 &=& \frac{9-3x_1-1x_2}{5}
\end{array}$$

Tomemos como condición inicial $X^0 = [x_1^0,x_2^0,x_2^0] = [0,0,0]$

\begin{minipage}{.65\linewidth}
\pause
$$\begin{array}{ccc}
x_1^1 = \frac{12-3*0-1*0}{8} &\rightarrow & \boxed{x_1^1 = 1.5}\\
x_2^1 = \frac{9-1*{\color{red}1.5}-3*0}{5} &\rightarrow & \boxed{x_2^1 = 1.5}\\
x_3^1 = \frac{9-3*{\color{red}1.5}-1*{\color{red}1.5}}{5} &\rightarrow & \boxed{x_3^1 = 0.6}
\end{array}$$

\end{minipage}\pause \vrule \begin{minipage}{.3\linewidth}
con lo que mi nuevo candidato será:

$$x = \begin{bmatrix}
1.5\\
1.5\\
0.6
\end{bmatrix}$$

\end{minipage}
\end{frame}

%paso 2
\begin{frame}
\frametitle{Ejemplo: Gauss-Seidel}

\begin{minipage}{.4\linewidth}
El problema:\vspace{-5pt}
$$\begin{array}{ccc}
x_1 &=& \frac{12-3x_2-1x_3}{8}\\
x_2 &=& \frac{9-1x_1-3x_3}{5}\\
x_3 &=& \frac{9-3x_1-1x_2}{5}
\end{array}$$

mi estado actual $X^1 =[1.5,1.5,0.6]$\vspace{-5pt}
\end{minipage} \vrule \begin{minipage}{.5\linewidth}
\begin{center}
  \textbf{Errores}
\end{center}
$$e_1^1 = \frac{|1.5 -0|}{1.5} = 1$$
$$e_2^1 = 1$$
$$e_3^1 = 1$$
\end{minipage}



\begin{minipage}{.65\linewidth}
\pause
$$\begin{array}{ccc}
x_1^2 = \frac{12-3*1.5-1*0.6}{8} &\rightarrow & \boxed{x_1^2 = 0.86}\\
x_2^2 = \frac{9-1*{\color{red}0.86}-3*0.6}{5} &\rightarrow & \boxed{x_2^2 = 1.26}\\
x_3^2 = \frac{9-3*{\color{red}0.86}-1*{\color{red}1.26}}{5} &\rightarrow & \boxed{x_3^2 = 1.02}
\end{array}$$

\end{minipage}\pause \vrule \begin{minipage}{.3\linewidth}
con lo que mi nuevo candidato será:

$$x = \begin{bmatrix}
0.86\\
1.26\\
1.02
\end{bmatrix}$$

\end{minipage}
\end{frame}


%paso 3
\begin{frame}
\frametitle{Ejemplo: Gauss-Seidel}

\begin{minipage}{.4\linewidth}
El problema:\vspace{-5pt}
$$\begin{array}{ccc}
x_1 &=& \frac{12-3x_2-1x_3}{8}\\
x_2 &=& \frac{9-1x_1-3x_3}{5}\\
x_3 &=& \frac{9-3x_1-1x_2}{5}
\end{array}$$

mi estado actual $X^2 =[0,86,1.26,1.02]$\vspace{-5pt}
\end{minipage} \vrule \begin{minipage}{.5\linewidth}
\begin{center}
  \textbf{Errores}
\end{center}$$e_1^2 = \frac{|0.86 -1.5|}{0.86} 	= 0.73$$
$$e_2^2 = 0.18$$
$$e_3^2 = 0.41$$
\end{minipage}



\begin{minipage}{.65\linewidth}
\pause
$$\begin{array}{ccc}
x_1^3 = \frac{12-3*1.26-1*1.02}{8} &\rightarrow & \boxed{x_1^3 = 0.89}\\
x_2^3 = \frac{9-1*{\color{red}0.89}-3*1.02}{5} &\rightarrow & \boxed{x_2^3 = 1.00}\\
x_3^3 = \frac{9-3*{\color{red}0.89}-1*{\color{red}1.0}}{5} &\rightarrow & \boxed{x_3^3 = 1.06}
\end{array}$$

\end{minipage}\pause \vrule \begin{minipage}{.3\linewidth}
con lo que mi nuevo candidato será:

$$x = \begin{bmatrix}
0.89\\
1.00\\
1.06
\end{bmatrix}$$

\end{minipage}
\end{frame}


%paso 4
\begin{frame}
\frametitle{Ejemplo: Gauss-Seidel}

\begin{minipage}{.4\linewidth}
El problema:\vspace{-5pt}
$$\begin{array}{ccc}
x_1 &=& \frac{12-3x_2-1x_3}{8}\\
x_2 &=& \frac{9-1x_1-3x_3}{5}\\
x_3 &=& \frac{9-3x_1-1x_2}{5}
\end{array}$$

mi estado actual $X^3 =[0,89,1.00,1.06]$\vspace{-5pt}
\end{minipage} \vrule \begin{minipage}{.5\linewidth}
\begin{center}
  \textbf{Errores}
\end{center}
$$e_1^3 = \frac{|0.89 -0.86|}{0.89} 	= 0.03$$
$$e_2^3 = 0.26$$
$$e_3^3 = 0.03$$
\end{minipage}

\begin{minipage}{.65\linewidth}
\pause
$$\begin{array}{ccc}
x_1^4 = \frac{12-3*1.00-1*1.02}{8} & \rightarrow & \boxed{x_1^4 = 0.99}\\
x_2^4 = \frac{9-1*{\color{red}0.99}-3*1.06}{5} &\rightarrow & \boxed{x_2^4 = 0.96}\\
x_3^4 = \frac{9-3*{\color{red}0.99}-1*{\color{red}0.96}}{5} &\rightarrow & \boxed{x_3^4 = 1.01}
\end{array}$$

\end{minipage}\pause \vrule \begin{minipage}{.3\linewidth}
con lo que mi nuevo candidato será:
$$x = \begin{bmatrix}
0.99\\
1.96\\
1.01
\end{bmatrix}$$
\end{minipage}

\end{frame}







\subsection{Convergencia de Gauss-Seidel}
\begin{frame}
\frametitle{Convergencia de Gauss-Seidel}

Puede demostrarse que el algoritmo de Gauss-Seidel es convergente si la matriz $A$ es  \textbf{diagonal dominante}.

\pause
\begin{tcolorbox}
\textbf{Definición:} se dice que una Matriz $A in \mathbb{R}^{n\times n}$ es diagonal dominante si cumple que:

$$\forall i,j=1,2,..n :\ \  |a_{i,i}| > \sum_{\substack{j=1 \\j\neq i}}^n  |a_{i,j} |$$
\end{tcolorbox}

\end{frame}

\begin{frame}
\frametitle{Convergencia de Gauss-Seidel}
Veamos que la matriz $A$ del ejemplo es Diagonal Dominante:

$$ |a_{1,1}| : 8 < |a_{1,2}| + |a_{1,3}| = 3 +1 $$
$$ |a_{2,2}| : 5 < 1 + 3 $$
$$ |a_{3,3}| : 5 < 1 + 3 $$

\pause
\begin{tcolorbox}
\textbf{IMPORTANTE:}
 Esta condición es suficiente pero no necesaria!
\end{tcolorbox}

\end{frame}

\subsection{Consideraciones}

\begin{frame}
\frametitle{Consideraciones}

\begin{itemize}
\item Es un método iterativo.
\item Sirve para aplicar en grandes sistemas.
\item Es posible de antemano saber su convergencia.
\end{itemize}
\end{frame}


\section{Método de Gauss-Seidel con relajación}

\begin{frame}
\frametitle{Gauss-Seidel con Relajación}
Se propone una modificación al algoritmo, para mejorar la convergencia:\vspace{15pt}


\pause
Se calcula el valor de una incógnita $x_i^j$ y se propone utilizar como candidato a valor un promedio ponderado con el valor anterior:
$$ \hat x_i^j = \alpha x_i^{j} + (1-\alpha) x_i^{j-1} $$

Donde $\alpha$ es un parámetro a definir.

\end{frame}

\subsection{Ejemplo Gauss-Seidel con relajación}


%Start
\begin{frame}
  \frametitle{Ejemplo: Gauss-Seidel Con relajación}
	Supongamos que tenemos el problema:
$$  \begin{bmatrix}
8 & 3 & 1 \\
1 & 5 & 3 \\
3 & 1 & 5
\end{bmatrix} X = \begin{bmatrix}
12\\
9\\
9
\end{bmatrix} \rightarrow \begin{array}{ccc}
x_1 &=& \frac{12-3x_2-1x_3}{8}\\
x_2 &=& \frac{9-1x_1-3x_3}{5}\\
x_3 &=& \frac{9-3x_1-1x_2}{5}
\end{array}$$

Tomemos $X^0 = [x_1^0,x_2^0,x_3^0] = [0,0,0]$ y $ \alpha =0.8 $

\begin{minipage}{.55\linewidth}
\pause
\small
$$\begin{array}{ccc}
x_1^1 = \frac{12-3*0-1*0}{8} &\rightarrow & \boxed{x_1^1 = 1.5}\\
x_2^1 = \frac{9-1*{\color{red}1.2}-3*0}{5} &\rightarrow & \boxed{x_2^1 = 1.56}\\
x_3^1 = \frac{9-3*{\color{red}1.2}-1*{\color{red}1.25}}{5} &\rightarrow & \boxed{x_3^1 = 0.83}
\end{array}$$

\end{minipage} \vrule \begin{minipage}{.4\linewidth}
con lo que mi nuevo candidato será:
\small
$$x = \begin{bmatrix}
1.5\\
1.56\\
0.83
\end{bmatrix} \alpha + (1-\alpha)\begin{bmatrix}
0\\
0\\
0
\end{bmatrix} = \begin{bmatrix}
1.2\\
1.25\\
0.66
\end{bmatrix}$$

\end{minipage}
\end{frame}



%paso 2
\begin{frame}
\frametitle{Ejemplo: Gauss-Seidel}

\begin{minipage}{.4\linewidth}
El problema:\vspace{-5pt}
$$\begin{array}{ccc}
x_1 &=& \frac{12-3x_2-1x_3}{8}\\
x_2 &=& \frac{9-1x_1-3x_3}{5}\\
x_3 &=& \frac{9-3x_1-1x_2}{5}
\end{array}$$

mi estado actual $X^1 =[1.2,1.25,0.66]$\vspace{-5pt}
\end{minipage} \vrule \begin{minipage}{.5\linewidth}
\begin{center}
  \textbf{Errores}
\end{center}
$$e_1^1 = \frac{|1.2 -0|}{1.2} = 1$$
$$e_2^1 = 1$$
$$e_3^1 = 1$$
\end{minipage}


\begin{minipage}{.55\linewidth}
\pause
\small
$$\begin{array}{cll}
x_1^2 = \frac{12-3*1.25-1*0.66}{8} &\rightarrow & \boxed{x_1^2 = 0.95}\\
x_2^2 = \frac{9-1*{\color{red}1.00}-3*0.66}{5} &\rightarrow & \boxed{x_2^2 = 1.20}\\
x_3^2 = \frac{9-3*{\color{red}1}-1*{\color{red}1.21}}{5} &\rightarrow & \boxed{x_3^2 = 0.96}
\end{array}$$

\end{minipage} \vrule \begin{minipage}{.35\linewidth}
\small
con lo que mi nuevo candidato será:
$$x = \begin{bmatrix}
0.95\\
1.20\\
0.96
\end{bmatrix} 0.8 +0.2\begin{bmatrix}
1.2\\
1.25\\
0.66
\end{bmatrix} = \begin{bmatrix}
1.00\\
1.21\\
0.90
\end{bmatrix}$$

\end{minipage}
\end{frame}




%paso 3
\begin{frame}
\frametitle{Ejemplo: Gauss-Seidel}

\begin{minipage}{.4\linewidth}
El problema:\vspace{-5pt}
$$\begin{array}{ccc}
x_1 &=& \frac{12-3x_2-1x_3}{8}\\
x_2 &=& \frac{9-1x_1-3x_3}{5}\\
x_3 &=& \frac{9-3x_1-1x_2}{5}
\end{array}$$

mi estado actual $X^2 =[1.00,1.21,0.90]$ y $\alpha=0.8$ \vspace{-5pt}
\end{minipage} \vrule \begin{minipage}{.5\linewidth}
\begin{center}
  \textbf{Errores}
\end{center}
$$e_1^2 = \frac{|1.00 -1.2|}{1.00} = 0.2$$
$$e_2^2 = 0.03$$
$$e_3^2 = 0.26$$
\end{minipage}


\begin{minipage}{.55\linewidth}
\pause
\small
$$\begin{array}{cll}
x_1^2 = \frac{12-3*1.21-1*0.9}{8} &\rightarrow & \boxed{x_1^2 = 0.93}\\
x_2^2 = \frac{9-1*{\color{red}0.95}-3*0.9}{5} &\rightarrow & \boxed{x_2^2 = 1.07}\\
x_3^2 = \frac{9-3*{\color{red}0.95}-1*{\color{red}1.10}}{5} &\rightarrow & \boxed{x_3^2 = 1.01}
\end{array}$$

\end{minipage} \vrule \begin{minipage}{.35\linewidth}
\small
con lo que mi nuevo candidato será:
$$x = \begin{bmatrix}
0.93\\
1.07\\
1.01
\end{bmatrix} 0.8 +0.2\begin{bmatrix}
1.00\\
1.21\\
0.90
\end{bmatrix} = \begin{bmatrix}
0.95\\
1.10\\
0.99
\end{bmatrix}$$

\end{minipage}
\end{frame}

\subsection{Elección de $\alpha$}

\begin{frame}
\frametitle{Elección de $\alpha$}
La elección es empírica entre $(0,2)$.


\begin{itemize}
\pause
\item Si $\alpha=1$ se trata de Gauss-Seidel Clásico.
\end{itemize}

\textbf{Típicamente:}

\begin{minipage}{.45\linewidth}
\pause
$\alpha \in (0,1)$: (sub-relajación) 

$\hat x^j \in (x^{j-1},x^j)$, es decir un promedio.
\pause 

Ayuda si el sistema original no converge.
\end{minipage}  \vrule  \begin{minipage}{.45\linewidth}
\pause
$\alpha \in (1,2)$ (sobre-relajación) 
$\hat x^j \notin (x^{j-1},x^j)$.\vspace{10pt}
\pause 

Ayuda si converge lento.
\end{minipage}

\end{frame}

\section{Sistemas de Ecuaciones no lineales}

\begin{frame}
\frametitle{Sistemas de Ecuaciones no lineales}
Es posible extender métodos vistos a SENL.\vspace{15pt}
\pause 

Cuidado, estos sistemas son más sensibles a condiciones iniciales.\vspace{15pt}
\pause

\textbf{Ejemplo}
Sea el sistema:
$$\begin{array}{ccc}
x_1^2 +x_1 x_2 &=& 10\\
x_2 +3x_1 x_2^2 &=& 57
\end{array}$$
tomando $X = [1,4]$ 

\end{frame}

\begin{frame}
\frametitle{SENL: ejemplo}
Podemos despejar:
$$\begin{array}{ccc}
x_1 & = & \sqrt{10-x_1 x_2} \\
x_2 & = & \sqrt{\frac{57-x_2}{3x_1}} \\
\end{array} $$
\pause
Si aplicamos Gauss-Seidel

\begin{minipage}{.45\linewidth}
$$\begin{array}{lcc}
x_1^1 = \sqrt{10 -1*4} & = & 2.44 \\
x_2^1 = \sqrt{ \frac{57 - 4}{3* {\color{red} 2.44} }} & = & 2.68
\end{array}  $$
\end{minipage} \vrule \begin{minipage}{.45\linewidth}
$$\begin{array}{lcc}
x_1^2 = \sqrt{10 -2.44*2.68} & = & 1.84 \\
x_2^2 = \sqrt{ \frac{57 - 2.68}{3* {\color{red} 1.84} }} & = & 3.12
\end{array}  $$
\end{minipage}

\end{frame}


\begin{frame}
\frametitle{SENL: ejemplo}
Si se sigue iterando se llegará a la solución $[x_1,x_2] = [2,3]$.

\pause
\textbf{Cuidado}, no siempre los SENL convergen, incluso esto puede variar dependiendo el despeje. Queda como tarea para casa, probar otro despeje de las ecuaciones y comprobar su divergencia.
\end{frame}


\end{document}








