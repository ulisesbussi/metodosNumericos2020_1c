
%----------------------------------------------------------------------
\documentclass[xcolor=svgnames]{beamer} %, handout
\usetheme{lined}
\usepackage{array}
\usepackage{amsmath}
\usepackage{amssymb}
\usepackage{amsthm}
\usepackage{bbm}
\usepackage[utf8]{inputenc}
\usepackage[T1]{fontenc}
\usepackage{cmbright}   
\usepackage{times}
\usepackage{geometry}
\usepackage[spanish]{babel}
\usepackage{multicol}
\usepackage{tcolorbox}

\usepackage{algorithm2e}

%\usepackage{psfrag}
\usepackage{graphicx}
\usepackage{color}
\usepackage{floatflt}
\usepackage{fancybox}
\usepackage{tabularx}
%\usepackage[all]{xy}
\usepackage{color}
\usepackage{siunitx} % para \degree



\usepackage{tikz}
\usetikzlibrary{positioning}




%\usepackage[most]{tcolorbox}


\theoremstyle{plain}
\newtheorem{definicion}{Definición}



\definecolor{redUnq}{rgb}{.7,.1,.1}
\definecolor{redUnq2}{rgb}{.5,.1,.3}

\mode<presentation>{
	%\usetheme{Boxes}
	%\usecolortheme[RGB={237,132,8}]{structure}
	%\usecolortheme[RGB={205,173,0}]{structure}
	\usecolortheme[RGB={100,10,10}]{structure}

	%\beamertemplateshadingbackground{SteelBlue!70}{Honeydew!10}
	%\usetheme{Warsaw}
	%\usecolortheme{default}
	\usetheme{Singapore}
	%\usetheme{Lined}
	%\usetheme[height=7mm]{Rochester} 
	\setbeamerfont{title}{shape=\bfseries,family=\rmfamily}
	%\usefonttheme[onlylarge]{structuresmallcapsserif}
	%\usefonttheme[onlysmall]{structurebold}
	\setbeamercolor{title}{fg=redUnq,bg=gray!40}
	\usefonttheme{professionalfonts}
	\setbeamercovered{highly dynamic}
	\setbeamercovered{transparent=10}
	\setbeamertemplate{navigation symbols}{}
	\colorlet{structure}{redUnq}

	\setbeamertemplate{frametitle}[default][left]
}

\definecolor{verzul}{rgb}{0, 0.5,0.5}

\renewcommand{\textbf}[1]{{\bfseries\textcolor{redUnq2}{#1}}}
\renewcommand{\emph}[1]{{\em\textcolor{redUnq2}{#1}}}

\setlength{\parindent}{0pt}
\theoremstyle{definition}
\newtheorem{ejem}{Ejemplo}
\newtheorem{defi}{Definición}
\newtheorem{ejer}{Ejercicio}
\newtheorem{prop}{Propiedad}
\newtheorem{lema}{Lema}
\newtheorem{teor}{Teorema}
\newtheorem{coro}{Corolario}

 

\newcommand{\Rset}{\mathbbmss{R}}
\newcommand{\Cset}{\mathbbmss{C}}
\newcommand{\PD}[2]{\frac{\partial #1}{\partial #2}}
\DeclareMathOperator{\tr}{tr}
\DeclareMathOperator{\adj}{adj}
\DeclareMathOperator{\rango}{rango}

\newenvironment{Boxedminipage}%
{\begin{Sbox}\begin{minipage}}%
{\end{minipage}\end{Sbox}\fbox{\TheSbox}}



\title{Métodos Numéricos - Clase 1}
  \logo{\includegraphics[scale=0.25]{logoUnq} }
\author{Ulises Bussi- Javier Portillo}
%\institute{\scalebox{2}{\includegraphics[scale=0.1]{mdp02.jpg}}} %{Departamento de Ciencia y Tecnología\\ Universidad Nacional de Quilmes\\ }
\date{ $1^\circ$ cuatrimestre 2020} 


%%%%%%%%% Para que al comenzar una section aparezca el Contenido
%\AtBeginSection[]
%{
%  \begin{frame}
%    \frametitle{Contenidos de la Presentación}
%    \tableofcontents[currentsection]
%  \end{frame}
%}




\begin{document} 


\begin{frame} %\thispagestyle{empty}
	\titlepage
\end{frame}


\begin{frame}
\frametitle{Antes de empezar}
\begin{tcolorbox}
	\begin{defi} \vspace{15pt}

		Un método numérico es un procedimiento que permite obtener, \textbf{de forma aproximada}, solución a un problema, por medio de la aplicación de algoritmos.
	\end{defi}
\end{tcolorbox}


\end{frame}

\begin{frame}
\frametitle{Nociones de Error}




%\begin{minipage}{.4\linewidth}
\textbf{¿Que es un error?}

\pause

\begin{tcolorbox}
Un error es todo aquello que hace que nuestra representación de un objeto de estudio difiera de la realidad
\end{tcolorbox}


\begin{center}

  \begin{tikzpicture}[
   squarednode/.style={rectangle, draw=red!60, fill=red!5, minimum size=.1mm},
   ]
   %Nodes
   \node[squarednode]	(errores)  																	{Errores Totales};
   \node[squarednode] (modelado) 			[below = of errores, xshift= -15mm]		{Modelado};
   \node[squarednode] (numerico) 			[below = of errores, xshift=  15mm]		{Numéricos};
   \node[squarednode] (truncamiento)	[below = of numerico, xshift=-15mm] 	{Truncamiento};
   \node[squarednode] (redondeo)     	[below = of numerico, xshift= 15mm]	{Redondeo};
 
   %Lines
   %\draw[->] (uppercircle.north) -- (errores.north);
   \draw[->] (errores.south) -- (modelado.north);
   \draw[->] (errores.south) -- (numerico.north);
   \draw[->] (numerico.south) -- (truncamiento.north);
   \draw[->] (numerico.south) -- (redondeo.north);
  \end{tikzpicture}
\end{center}

%\end{minipage}

	
\end{frame}


\begin{frame}
\frametitle{Errores Numéricos}
\textbf{Error de truncamiento}: expresar con menor cantidad de cifras significativas un número. (e.g. aproximar el valor de una función usando un polinomio de taylor de un grado determinado).
\vspace{10pt}


\textbf{Error de redondeo}: Dado por la incapacidad de expresar números infinitos en una computadora (e.g. expresar un numero irracional numéricamente)


\vspace{10pt}
\pause

\begin{tcolorbox}
\textbf{IMPORTANTE:}
Estos errores se pueden acumular en los métodos numéricos e incluso amplificar
\end{tcolorbox}

\end{frame}



\begin{frame}
\frametitle{Representación de errores}
Existen varias maneras, las más comunes:

\begin{minipage}{.45\linewidth}
  \textbf{Error Absoluto:}

  $$ e_a = | \hat x - x |$$
  \vspace{3pt}
  
  \underline{e.g.:}
  $$x = 280 \si{\degree}K\ ,\ \hat x = 273 \si{\degree}K$$
	Error absoluto:
	$ 7 \si{\degree}K$
\end{minipage} \vline \hspace{10pt}
\begin{minipage}{.45\linewidth}
	\vspace{10pt}
  \textbf{Error relativo:}

  $$ e_r = \frac{| \hat x - x |}{|x|}$$

  \underline{e.g.:}
  $$x = 350 \si{\degree}K\ ,\ \hat x = 273 \si{\degree}K$$
	Error relativo:
	$ 0.025 \approx 2.5\%$
\end{minipage}\vspace{10pt}


Cada uno aporta cierta información sobre el error, ninguna es incorrecta.


\end{frame}



\begin{frame}
\frametitle{Errores Numéricos}

\textbf{Example time!}


{\color{red} caida libre con redondeo }\vspace{10pt}


{\color{red} calcular el valor de $\sqrt{2}$ con un polinomio de taylor}\vspace{10pt}

La importancia de estos errores se apreciará mejor cuando se presenten métodos iterativos.
\end{frame}


\begin{frame}
	\frametitle{Convergencia}

	Método numérico $\rightarrow$ conjunto de parámetros propios.
	
	(e.g. en polinomio de Taylor el orden del mismo)\vspace{10pt}
	
	Estos determinan:
\begin{itemize}
	\item Comportamiento.
	\item Exigencia.
	\item Calidad de la solución.
\end{itemize}
	 
	 
**\textit{Poor Definition}**\\
La convergencia de un algoritmo relaciona, como al aumentar la exigencia aumenta la calidad de la solución. 

\textbf{Depende no solo del algoritmo, sino también del problema a resolver.}

\end{frame}


\begin{frame}
\frametitle{Estabilidad}
Cuando un método depende de parámetros de entrada (veremos más adelante), \\
la estabilidad del método relaciona la variación de los parametros de entrada con el resultado obtenido.


\end{frame}







\end{document}

%%% Local Variables: 
%%% mode: latex
%%% TeX-master: t
%%% End: 







