\documentclass[a4paper,11pt]{article}

\usepackage[T1]{fontenc}
\usepackage[utf8]{inputenc}
\usepackage{graphicx}
\usepackage{xcolor}
\usepackage{tcolorbox}
\usepackage{sectsty}
\usepackage{enumitem}
\usepackage{circuitikz}
\renewcommand\familydefault{\sfdefault}
\usepackage{tgheros}
\usepackage{amsmath,amssymb,amsthm,textcomp}
\usepackage{enumitem}
\usepackage{multicol}
\usepackage{tikz}

\usepackage{geometry}
\geometry{left=25mm,right=25mm,%
bindingoffset=0mm, top=20mm,bottom=20mm}





\linespread{1.3}

\newcommand{\linia}{\rule{\linewidth}{0.5pt}}

% custom theorems if needed
\newtheoremstyle{mytheor}
    {1ex}{1ex}{\normalfont}{0pt}{\scshape}{.}{1ex}
    {{\thmname{#1 }}{\thmnumber{#2}}{\thmnote{ (#3)}}}

\theoremstyle{mytheor}
\newtheorem{defi}{Definition}

% my own titles
\makeatletter

\renewcommand{\maketitle}{
\colorbox{gray!20}{\framebox[\linewidth]{ \huge \textsc{\@title} } 
\lfoot{\@title}
}

}
\makeatother
%%%



% custom footers and headers
\usepackage{fancyhdr}
\pagestyle{fancy}
\lhead{}
\chead{Métodos Numéricos - 1C2020}
\rhead{}
\cfoot{}
\rfoot{Pagina \thepage}
\renewcommand{\footrulewidth}{0.2pt}
%


\renewcommand\headrulewidth{.5pt}
\makeatletter
\def\headrule{{\if@fancyplain\let\headrulewidth\plainheadrulewidth\fi
\hrule\@height\headrulewidth\@width\headwidth
\vskip 1pt% 2pt between lines
\hrule\@height1.5pt\@width\headwidth% lower line with .5pt line width
\vskip-\headrulewidth
\vskip-1.5pt}}
\makeatother


\makeatletter

\newcommand{\inlineitem}[1][]{%
\ifnum\enit@type=\tw@
    {\descriptionlabel{#1}}
  \hspace{\labelsep}
\else
  \ifnum\enit@type=\z@
       \refstepcounter{\@listctr}\fi
    \quad\@itemlabel\hspace{\labelsep}
\fi}
\makeatother




% code listing settings
\usepackage{listings}
\lstset{
    language=Python,
    basicstyle=\ttfamily\small,
    aboveskip={1.0\baselineskip},
    belowskip={1.0\baselineskip},
    columns=fixed,
    extendedchars=true,
    breaklines=true,
    tabsize=4,
    prebreak=\raisebox{0ex}[0ex][0ex]{\ensuremath{\hookleftarrow}},
    frame=lines,
    showtabs=false,
    showspaces=false,
    showstringspaces=false,
    keywordstyle=\color[rgb]{0.627,0.126,0.941},
    commentstyle=\color[rgb]{0.133,0.545,0.133},
    stringstyle=\color[rgb]{01,0,0},
    numbers=left,
    numberstyle=\small,
    stepnumber=1,
    numbersep=10pt,
    captionpos=t,
    escapeinside={\%*}{*)}
}

%%%----------%%%----------%%%----------%%%----------%%%

\begin{document}


\title{Trabajo práctico 3}

\author{Ulises Bussi-Javier Portillo, Universidad Nacional de Quilmes}


\maketitle \vspace{20pt}

\section*{Ajustes e interpolación}

\subsection*{Ejercicio 2}
La siguiente tabla contiene un conjunto de puntos por los que tiene que pasar
un robot móvil en su trayectoria. para unir dichos puntos se suele utilizar
un interpolador que permita unir el conjunto de puntos. 

\begin{table}[h!]
\centering
\begin{tabular}{l|llllllll}
 \hline
 tiempo (s)   & 1	 &  2    &  3   &  4    &  5    &   6   &   7   &  8     \\ \hline
 x (cm) 	  & 3    &  6    &  11  &  18   & 13    &   9   &   5   &  2    \\ \hline
 y (cm) 	  & 3    &  5    &  7  &  10   & 14    &   25   &  20   &  25    \\ \hline
\end{tabular}
\end{table}


\begin{enumerate}[label=\alph*)]
  \item Realice una interpolación con spline de grado 2, para cada dimensión.
  \item Grafique $x$ en función de $t$, e $y$ en función de $t$.
  \item Grafique la trayectoria  ( $x$ vs $y$ ).
  \item Interprete ¿Por qué utilizar spline de grado 2 y no una interpolación lineal?
\end{enumerate} 





\section*{Derivadas e integrales}
%

\subsection*{Ejercicio 3} 

Los datos de la función "ej3TP3" (llamela [t,x,y] = ej3TP3() )contienen puntos sobre una circunferencia de radio 1.

\begin{enumerate}[label=\alph*)]
  \item Calcule la derivada hacia adelante.
  \item Calcule la derivada centrada.
  \item Compruebe para ambos casos si el gradiente es perpendicular a la curva en algún punto dado. En caso de que no se cumpla esa condición, explique ¿A que podría deverse?

\end{enumerate} 







\subsection*{Ejercicio 4} 

Evalue la siguiente integral:

\begin{equation*}
    \int_{0}^{\pi}sin(3x)dx
\end{equation*}

\begin{enumerate}[label=\alph*)]
    \item Analíticamente
    \item Usando los métodos: del trapecio, punto medio, Simpson 1/3 y Simpson 3/8 con item n=[1,2,3,4,5,6,7,8,9,10]
    \item Calcule el error relativo para cada metodo y cada n. Grafique.
    \item Analice y explique los valores obtenidos en el punto c)
\end{enumerate} 


\end{document}