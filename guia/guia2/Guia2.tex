

\documentclass[a4paper,11pt]{article}

\usepackage[T1]{fontenc}
\usepackage[utf8]{inputenc}
\usepackage{graphicx}
\usepackage{xcolor}
\usepackage{tcolorbox}
\usepackage{sectsty}
\usepackage{enumitem}
\usepackage{circuitikz}
\renewcommand\familydefault{\sfdefault}
\usepackage{tgheros}
\usepackage{amsmath,amssymb,amsthm,textcomp}
\usepackage{enumitem}
\usepackage{multicol}
\usepackage{tikz}

\usepackage{geometry}
\geometry{left=25mm,right=25mm,%
bindingoffset=0mm, top=20mm,bottom=20mm}





\linespread{1.3}

\newcommand{\linia}{\rule{\linewidth}{0.5pt}}

% custom theorems if needed
\newtheoremstyle{mytheor}
    {1ex}{1ex}{\normalfont}{0pt}{\scshape}{.}{1ex}
    {{\thmname{#1 }}{\thmnumber{#2}}{\thmnote{ (#3)}}}

\theoremstyle{mytheor}
\newtheorem{defi}{Definition}

% my own titles
\makeatletter

\renewcommand{\maketitle}{
\colorbox{gray!20}{\framebox[\linewidth]{ \huge \textsc{\@title} } 
\lfoot{\@title}
}

}
\makeatother
%%%



% custom footers and headers
\usepackage{fancyhdr}
\pagestyle{fancy}
\lhead{}
\chead{Métodos Numéricos - 1C2020}
\rhead{}
\cfoot{}
\rfoot{Pagina \thepage}
\renewcommand{\footrulewidth}{0.2pt}
%


\renewcommand\headrulewidth{.5pt}
\makeatletter
\def\headrule{{\if@fancyplain\let\headrulewidth\plainheadrulewidth\fi
\hrule\@height\headrulewidth\@width\headwidth
\vskip 1pt% 2pt between lines
\hrule\@height1.5pt\@width\headwidth% lower line with .5pt line width
\vskip-\headrulewidth
\vskip-1.5pt}}
\makeatother


\makeatletter

\newcommand{\inlineitem}[1][]{%
\ifnum\enit@type=\tw@
    {\descriptionlabel{#1}}
  \hspace{\labelsep}
\else
  \ifnum\enit@type=\z@
       \refstepcounter{\@listctr}\fi
    \quad\@itemlabel\hspace{\labelsep}
\fi}
\makeatother




% code listing settings
\usepackage{listings}
\lstset{
    language=Python,
    basicstyle=\ttfamily\small,
    aboveskip={1.0\baselineskip},
    belowskip={1.0\baselineskip},
    columns=fixed,
    extendedchars=true,
    breaklines=true,
    tabsize=4,
    prebreak=\raisebox{0ex}[0ex][0ex]{\ensuremath{\hookleftarrow}},
    frame=lines,
    showtabs=false,
    showspaces=false,
    showstringspaces=false,
    keywordstyle=\color[rgb]{0.627,0.126,0.941},
    commentstyle=\color[rgb]{0.133,0.545,0.133},
    stringstyle=\color[rgb]{01,0,0},
    numbers=left,
    numberstyle=\small,
    stepnumber=1,
    numbersep=10pt,
    captionpos=t,
    escapeinside={\%*}{*)}
}

%%%----------%%%----------%%%----------%%%----------%%%

\begin{document}


\title{Guía 2}

\author{Ulises Bussi-Javier Portillo, Universidad Nacional de Quilmes}

\date{01/01/2014}

\maketitle \vspace{20pt}

\section*{Métodos Abiertos}
%

\subsection*{Algoritmos}

\begin{enumerate}[label=\Roman*]

\item \label{enum:fixedPoint} Cree una función que permita calcular raíces de una función utilizando el método del punto fijo, y tome como parámetros de entrada la función a analizar, la condición inicial y número de iteraciones a realizar.

\item \label{enum:newtonRaphson} Cree una función que permita calcular raíces de una función utilizando el método de Newton-Raphson, y tome como parámetros de entrada la función a analizar, la condición inicial y número de iteraciones a realizar.

\item \label{enum:secante} Cree una función que permita calcular raíces de una función utilizando el método de la secante, y tome como parámetros de entrada la función a analizar, la condiciones inicial y número de iteraciones a realizar.


\item \label{enum:errStop} Cree una nueva función que llame a \ref{enum:fixedPoint},\ref{enum:newtonRaphson},\ref{enum:secante} y permita además establecer un criterio de parada por error absoluto y relativo.


\end{enumerate}
\subsubsection*{Ejercicio 1}

Encuentre las raices de las siguientes funciones:

\begin{enumerate}	
\item $f(x) = 3x^2 + 5x -6$ con  $x_0 \in [1,5]$
\item $g(x) = 0.3x^2 -cos(x)$ con $x_0 \in [0,2]$
\item $h(x) = cos(x^2-1)e^{\sqrt[3]{\sin x  (x^2 -x +1)}}\log{(x^6+\frac{x}{x^2-1})}$ con $x_0 \in [1,2]$

\end{enumerate}

\begin{itemize}
\item Utilizando el método de la \textbf{punto fijo}, con un error relativo de $1\%$, ¿puede asegurar de antemando la convergencia o divergencia?
\item Utilizando el método de Newton-Raphson y método de la secante. Compare el número de iteraciones utilizado por cada método para llegar a valores similares de error. ¿Qué método utilizaría en cada caso? por qué?

\end{itemize}


\subsubsection*{Ejercicio 2}

Utilice el método de Newton-Raphson para determianr la raiz real de 
$f(x) = -2+6x-4x^2+0.5 x^3$, usando como condición inicial
\textbf{(a)} 4.5 y \textbf{(b)} 4.43. ¿Qué sucede? ¿Cómo podría explicarlo?

\subsubsection*{Ejercicio 3}

La fórmula de dividir y promediar, es un método de aproximación de la raiz cuadrada 
de un número positivo real $a$. Este método se basa en calcular el candidato a solución
$$ x_{i+1} = \frac{x_i+ a/x_i}{2}$$
Pruebe que puede llegarse a esta fórmula partiendo del método de Newton-Raphson.

\subsubsection*{Ejercicio 4}

La impedancia ($Z$) de un circuito RLC en paralelo , como el de la figura \ref{fig:elec},puede calcularse como:
$$\frac{1}{Z} = \sqrt{\frac{1}{R^2}+ 1(\omega C - \frac{1}{\omega L})^2}$$

encuentre la frecuencia angular $\omega$ que hace que la impedancia $Z=100\Omega$ utilizando como condición inicial $\omega = 1$ y $\omega = 1000$, utilice el método de la secante, y la función propia de matlab fzero.

\begin{figure}[h!]
    \centering
    \begin{circuitikz} \draw
        (0,0) to[sinusoidal voltage source, o-o] (0,3) 
        (3,0) to[R,label=\mbox{$R$}, o-o]   (3,3)
        (6,0) to[L,label=\mbox{$L$}, o-o]   (6,3)
        (9,0) to[C,label=\mbox{$C$}, o-o]   (9,3)
        (0,0) -- (3,0) -- (6,0) --(9,0)
        (0,3)-- (3,3) -- (6,3) --(9,3)
        ;
    \end{circuitikz}
	\label{fig:elec}    
    \caption{Circuito ejercicio 4. }
    
\end{figure}


\subsubsection*{Ejercicio 5}

¿Cuáles son las diferencias entre el método de la Secante y el método de la falsa posición?.



\end{document}