

\documentclass[a4paper,11pt]{article}

\usepackage[T1]{fontenc}
\usepackage[utf8]{inputenc}
\usepackage{graphicx}
\usepackage{xcolor}
\usepackage{tcolorbox}
\usepackage{sectsty}
\usepackage{enumitem}

\renewcommand\familydefault{\sfdefault}
\usepackage{tgheros}
\usepackage{amsmath,amssymb,amsthm,textcomp}
\usepackage{enumitem}
\usepackage{multicol}
\usepackage{tikz}

\usepackage{geometry}
\geometry{left=25mm,right=25mm,%
bindingoffset=0mm, top=20mm,bottom=20mm}





\linespread{1.3}

\newcommand{\linia}{\rule{\linewidth}{0.5pt}}

% custom theorems if needed
\newtheoremstyle{mytheor}
    {1ex}{1ex}{\normalfont}{0pt}{\scshape}{.}{1ex}
    {{\thmname{#1 }}{\thmnumber{#2}}{\thmnote{ (#3)}}}

\theoremstyle{mytheor}
\newtheorem{defi}{Definition}

% my own titles
\makeatletter

\renewcommand{\maketitle}{
\colorbox{gray!20}{\framebox[\linewidth]{ \huge \textsc{\@title} } 
\lfoot{\@title}
}

}
\makeatother
%%%



% custom footers and headers
\usepackage{fancyhdr}
\pagestyle{fancy}
\lhead{}
\chead{Métodos Numéricos - 1C2020}
\rhead{}
\cfoot{}
\rfoot{Pagina \thepage}
\renewcommand{\footrulewidth}{0.2pt}
%


\renewcommand\headrulewidth{.5pt}
\makeatletter
\def\headrule{{\if@fancyplain\let\headrulewidth\plainheadrulewidth\fi
\hrule\@height\headrulewidth\@width\headwidth
\vskip 1pt% 2pt between lines
\hrule\@height1.5pt\@width\headwidth% lower line with .5pt line width
\vskip-\headrulewidth
\vskip-1.5pt}}
\makeatother


\makeatletter

\newcommand{\inlineitem}[1][]{%
\ifnum\enit@type=\tw@
    {\descriptionlabel{#1}}
  \hspace{\labelsep}
\else
  \ifnum\enit@type=\z@
       \refstepcounter{\@listctr}\fi
    \quad\@itemlabel\hspace{\labelsep}
\fi}
\makeatother




% code listing settings
\usepackage{listings}
\lstset{
    language=Python,
    basicstyle=\ttfamily\small,
    aboveskip={1.0\baselineskip},
    belowskip={1.0\baselineskip},
    columns=fixed,
    extendedchars=true,
    breaklines=true,
    tabsize=4,
    prebreak=\raisebox{0ex}[0ex][0ex]{\ensuremath{\hookleftarrow}},
    frame=lines,
    showtabs=false,
    showspaces=false,
    showstringspaces=false,
    keywordstyle=\color[rgb]{0.627,0.126,0.941},
    commentstyle=\color[rgb]{0.133,0.545,0.133},
    stringstyle=\color[rgb]{01,0,0},
    numbers=left,
    numberstyle=\small,
    stepnumber=1,
    numbersep=10pt,
    captionpos=t,
    escapeinside={\%*}{*)}
}

%%%----------%%%----------%%%----------%%%----------%%%

\begin{document}


\title{Guía 1}

\author{Ulises Bussi, Universidad Nacional de Quilmes}

\date{01/01/2014}

\maketitle \vspace{20pt}

\section*{Bracketing methods}
%
En principio esta guía está desarrollada para generar código que permita utilizar los métodos vistos en clase y adquirir habilidad en la programación. La idea principal es que les quede ya programados los métodos como funciones. 





\subsection*{Algoritmos}

\begin{enumerate}[label=\Roman*]

\item \label{enum:bisec} Cree una función que permita calcular raices de una función utilizando el método de la bisección, y tome como parámetros de entrada la función a analizar, el intervalo, número de iteraciones a realizar.

\item \label{enum:falseP} Cree una función que permita calcular raices de una función utilizando el método de la falsa posición, y tome como parámetros de entrada la función a analizar, el intervalo, número de iteraciones a realizar.


\item \label{enum:errStop} Cree una nueva función que llame a \ref{enum:bisec} y permita además establecer un criterio de parada por error absoluto y relativo.

\item Idem \ref{enum:errStop} pero llamando a \ref{enum:falseP} .

\end{enumerate}
\subsubsection*{Ejercicio 1}

Encuentre las raices de las siguientes funciones:

\begin{enumerate}
\begin{minipage}{.6\linewidth}
\item $f(x) = 3x^2 + 5x -6$ entre $[-5,-1]$ y $[-1,5]$
\item $g(x) = 0.3x^2 -cos(x)$ en $[0,2]$
\end{minipage}\begin{minipage}{.4\linewidth}
\item $h(x) = x\cos(x)$ en $[1,500]$
\item $i(x) = e^{-x^2} -e^{x/5}+2$ en $[0,6]$
\end{minipage}
$$j(x) = x^2 -2x +1, \text{ en } [0,3]$$


\begin{itemize}
\item Utilizando el método de la bisección, con un error relativo de $1\%$
\item Utilizando el método de la falsa posición, que coincida en los primeros dos decimales después de la coma con el método anterior.  
\item Grafique las funciones en el intervalo dado.
\end{itemize}
En ambos casos cuente el número de iteraciones y comparelos. ¿Que sucede con $j(x)$?

\end{enumerate}

\subsubsection*{Ejercicio 2}

La velocidad de caida de un paracaidista está dada por:
$$\displaystyle v = \frac{gm}{c} (1 - e^{ -(c/m) t})$$
Donde $g=9.81 m/s^2$ es la gravedad, el coeficiente de arrastre $c$ vale $15kg/s$ si se sabe que $v= 36 m/s$ a tiempo $t=10$ calcule la masa del paracaidista, utilizando el método de la falsa posición tal que el error relativo sea menor que $e_r < 0.1\%$


\subsubsection*{Ejercicio 3}

La curva de temperatura de un tanque de maceración industrial de cerveza en función del tiempo viene dada por la ecuación:
$$ T(t) = 100 - 80e^{t/3000}$$
;ientras la entrega de combustible está habilitada. Para una maceración exitosa se propone llevar el mosto a una temperatura de de $67^\circ C$ . ¿Cuanto tiempo lleva el calentamiento hasta la temperatura ideal? Calcule utilizando el método de bisección, con un error absoluto de $0.1^\circ C$

\subsubsection*{Ejercicio 4}

De acuerdo al principio de flotabilidad de Arquímedes, el empuje generado por un cuerpo es igual al peso del volumen desplazado del fluido. Si se tiene una esfera de radio $1m$ y densidad $\rho = 200kg/m^3$ que se sumerge parcialmente en agua (de densidad $\rho_{agua}=1000kg/m^3$. Calcule la altura h de de la esfera sobresale del agua  cuando la misma se encuentra en equilibrio. Establezca el método y criterio de error a utilizar.

Ayuda: se puede calcular el el volumen por arriba del agua como:
$V = \frac{\pi h^2}{3} (3r-h)$











\end{document}