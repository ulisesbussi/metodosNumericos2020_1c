\documentclass[a4paper, 11pt]{article}
\usepackage{comment} % enables the use of multi-line comments (\ifx \fi) 
\usepackage{fullpage} % changes the margin
\usepackage[spanish]{babel}
\selectlanguage{spanish}
\usepackage[utf8]{inputenc}
\usepackage{graphicx}
\usepackage{enumitem}
\usepackage{amsmath}
 

\begin{document}
\renewcommand{\listtablename}{Índice de tablas} 
\renewcommand{\tablename}{Tabla}
\begin{center}
\LARGE \bf Trabajo práctico número 2
\end{center}

\vspace{1cm} 
\noindent\textbf{Materia:} M\'etodos Num\'ericos \hfill
\textbf{Año} 2020 - 1C \\
\hfill \\
%\textbf{Fecha} \today \hfill \textbf{Ayudante:} Diego Hidalgo  \\ 
%\vspace{1cm} 
\section*{Ceros de funciones}

\subsection*{Ejercicio 1}
Utilizando su criterio, aplique un método numérico para encontrar los ceros (entre $0$ y $10$) de la función:
$$f(x) = \left[ e^{2x+2} - e^{3x+\ln(x+1)}\right]\left[ \cos(x+1) - \ln(3x^3 +3x+2)+ t\right] $$
Con un error estimado $e_r <0.1\%$.

\subsection*{Ejercicio 2}
Programe una función que tome como entrada una función anónima, una condición inicial, y una tolerancia; calcule el método de la secante y devuelva un vector con todos los candidatos hallados. 

\subsection*{Ejercicio 3}
Dada una función $g(x) = \frac{x}{7}^{10} -3$ ¿Qué desventajas presentaría aplicar el método de la falsa posición en el intervalo $[0,10]$? ¿Y el método de Newton-Raphson tomando como condición inicial $x_0 = 0$? ¿Que sucede con $x_0 =1$? ¿Cómo resolvería el problema? resuelvalo.

\section*{Sistemas de ecuaciones lineales}

\subsection*{Ejercicio 4}


\begin{enumerate}[label=\alph*]
\item- Cree una función que permita resolver el problema $A x=b$ utilizando Gauss-Jordan.
\item- Calcule el tiempo que tarda el algoritmo (tiempo de ejecución) en resolver $A x=b$ cuando A tiene dimensión: 20,40,...,480,500.
\item- Grafique: tiempo de ejecución versus dimensión.
\end{enumerate}

\subsection*{Ejercicio 5}

Una matriz tri-diagonal, es una matriz cuyos elementos son 0, a excepción de los elementos de la diagonal principal, y las 2 diagonales adyacentes a esta. Es decir:

\begin{equation} \label{myeq}
T_N = \left[ \begin{matrix}
a_{1,1} & a_{1,2} & \hdots & 0\\
a_{2,1} & \ddots & \ddots &  \vdots\\
\vdots & \ddots & \ddots & a_{n-1,n} \\
0 & \hdots & a_{n,n-1} & a_{n,n} \end{matrix} \right]
\end{equation}


\begin{enumerate}[label=\alph*]
\item- ¿Es posible aplicar la función creada en el problema 4-a si la matriz A es una matriz tri-diagonal? ¿El tiempo de ejecución cambiaría? Justifique. 
\item- Cree una función que genere matrices tri-diagonales de dimensión \textit{n}

\end{document}

© 2020 GitHub, Inc.
Terms
Privacy
Security
Status
Help
Contact GitHub
Pricing
API
Training
Blog
About
