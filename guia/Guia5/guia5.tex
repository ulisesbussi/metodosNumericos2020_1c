\documentclass[a4paper,11pt]{article}

\usepackage[T1]{fontenc}
\usepackage[utf8]{inputenc}
\usepackage{graphicx}
\usepackage{xcolor}
\usepackage{tcolorbox}
\usepackage{sectsty}
\usepackage{enumitem}
\usepackage{circuitikz}
\renewcommand\familydefault{\sfdefault}
\usepackage{tgheros}
\usepackage{amsmath,amssymb,amsthm,textcomp}
\usepackage{enumitem}
\usepackage{multicol}
\usepackage{tikz}
%\usepackage[table,xcdraw]{xcolor}

\usepackage{geometry}
\geometry{left=25mm,right=25mm,%
bindingoffset=0mm, top=20mm,bottom=20mm}





\linespread{1.3}

\newcommand{\linia}{\rule{\linewidth}{0.5pt}}

% custom theorems if needed
\newtheoremstyle{mytheor}
    {1ex}{1ex}{\normalfont}{0pt}{\scshape}{.}{1ex}
    {{\thmname{#1 }}{\thmnumber{#2}}{\thmnote{ (#3)}}}

\theoremstyle{mytheor}
\newtheorem{defi}{Definition}

% my own titles
\makeatletter

\renewcommand{\maketitle}{
\colorbox{gray!20}{\framebox[\linewidth]{ \huge \textsc{\@title} } 
\lfoot{\@title}
}

}
\makeatother
%%%



% custom footers and headers
\usepackage{fancyhdr}
\pagestyle{fancy}
\lhead{}
\chead{Métodos Numéricos - 1C2020}
\rhead{}
\cfoot{}
\rfoot{Pagina \thepage}
\renewcommand{\footrulewidth}{0.2pt}
%


\renewcommand\headrulewidth{.5pt}
\makeatletter
\def\headrule{{\if@fancyplain\let\headrulewidth\plainheadrulewidth\fi
\hrule\@height\headrulewidth\@width\headwidth
\vskip 1pt% 2pt between lines
\hrule\@height1.5pt\@width\headwidth% lower line with .5pt line width
\vskip-\headrulewidth
\vskip-1.5pt}}
\makeatother


\makeatletter

\newcommand{\inlineitem}[1][]{%
\ifnum\enit@type=\tw@
    {\descriptionlabel{#1}}
  \hspace{\labelsep}
\else
  \ifnum\enit@type=\z@
       \refstepcounter{\@listctr}\fi
    \quad\@itemlabel\hspace{\labelsep}
\fi}
\makeatother




% code listing settings
\usepackage{listings}
\lstset{
    language=Python,
    basicstyle=\ttfamily\small,
    aboveskip={1.0\baselineskip},
    belowskip={1.0\baselineskip},
    columns=fixed,
    extendedchars=true,
    breaklines=true,
    tabsize=4,
    prebreak=\raisebox{0ex}[0ex][0ex]{\ensuremath{\hookleftarrow}},
    frame=lines,
    showtabs=false,
    showspaces=false,
    showstringspaces=false,
    keywordstyle=\color[rgb]{0.627,0.126,0.941},
    commentstyle=\color[rgb]{0.133,0.545,0.133},
    stringstyle=\color[rgb]{01,0,0},
    numbers=left,
    numberstyle=\small,
    stepnumber=1,
    numbersep=10pt,
    captionpos=t,
    escapeinside={\%*}{*)}
}
 
%%%----------%%%----------%%%----------%%%----------%%%

\begin{document}


\title{Guía 5}

\author{Ulises Bussi-Javier Portillo, Universidad Nacional de Quilmes}


\maketitle \vspace{20pt}

\section*{Interpolación}


\subsection*{Algoritmos}

\begin{enumerate}[label=\Roman*]

\item Programe una función que le permita calcular el polinomio interpolador de Newton hasta grado 2.

\item Programe una función que le permita calcular el polinomio interpolador de Lagrange hasta grado n. Y retorne una función anónima para poder evaluar ese polinomio.

\item Programe una función que le permita interpolar con Splines de orden 1 o 2.

\end{enumerate}


\subsection*{Practica}

\begin{enumerate}
    \item Usar polinomios de Newton de orden 1,2 y 3 para calcular f(3.6).
        \begin{table}[h!]
        \centering
        \begin{tabular}{l|llllllll}
        \hline
        x & 0    & 0.5  & 1    & 1.5  & 2    & 3    & 4   & 5    \\ \hline
        f(x) & 3.26 & 2.09 & 1.62 & 1.48 & 1.17 & 1.06 & 0.9 & 0.85 \\ \hline
        \end{tabular}
        %\caption{}
        %\label{tab:my-table}
        \end{table}
        \begin{enumerate}
            \item ¿Que diferencia porcentual observa entre los distintos métodos utilizados?
            \item ¿Con qué resultado (entre los 3 calculados) se quedaría?
        \end{enumerate}
   
    
    \item Repita el punto 1), utilizando polinomios de Lagrange.



\item Usted esta dirigiendo una investigación que evalúa la posibilidad de la existencia de vida en Ganimedes, el conocido satélite de Júpiter. En particular, su interes esta en el lago Kierski, del cual se conocen los siguientes valores de concentración de oxigeno:

        \begin{table}[h!]
        \centering
        \begin{tabular}{l|lllllll}
        \hline
        $T[ ^\circ C]$ & 0    & 8  & 16    & 24  & 32    & 40   \\ \hline
        concentración [mg/L] & 14.621 & 11.843 & 9.870 & 8.418 & 7.305 & 6.413 \\ \hline
        \end{tabular}
        %\caption{}
        %\label{tab:my-table}
        \end{table}

        \begin{enumerate}
            \item Calcule la concentración a 28 grados utilizando splines.
            \item ¿Podría obtener ese valor utilizando una regresión? ¿Que ventajas o desventajas tiene cada caso?
        \end{enumerate}

\item En la robótica, para realizar el cálculo de trayectorias de manipuladores robóticos, se proponen puntos de interés y se calculan polinomios que pasen por dichos puntos (interpolación).

Realice una interpolación para generar una trayectoria que pase por los siguientes puntos para los tiempos dados:

      \begin{table}[h!]
        \centering
        \begin{tabular}{l|llllllll}
        \hline
        $t$ & 0    & 0.5  & 1    & 1.5  & 2  & 2.5  & 3   \\ \hline
        $x$ & 1    & 2    & 3    & 4    & 5  & 6    & 7   \\ \hline
		$y$ & 1    & 4    & 3    & 5    & 5  & 2    & 5   \\ \hline  
        \end{tabular}
        %\caption{}
        %\label{tab:my-table}
        \end{table}

     \begin{enumerate}
            \item Interpole para cada coordenada (para $x$ y para $y$).
            \item Realice un gráfico de cada coordenada en función del tiempo, desde la interpolación con un paso $0.1$.
            \item Realice un gráfico de la trayectoria ($x$ vs. $y$).
        \end{enumerate}


\end{enumerate}
\end{document}


