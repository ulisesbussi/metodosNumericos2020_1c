\documentclass[a4paper,11pt]{article}

\usepackage[T1]{fontenc}
\usepackage[utf8]{inputenc}
\usepackage{graphicx}
\usepackage{xcolor}
\usepackage{tcolorbox}
\usepackage{sectsty}
\usepackage{enumitem}
\usepackage{circuitikz}
\renewcommand\familydefault{\sfdefault}
\usepackage{tgheros}
\usepackage{amsmath,amssymb,amsthm,textcomp}
\usepackage{enumitem}
\usepackage{multicol}
\usepackage{tikz}
%\usepackage[table,xcdraw]{xcolor}

\usepackage{geometry}
\geometry{left=25mm,right=25mm,%
bindingoffset=0mm, top=20mm,bottom=20mm}





\linespread{1.3}

\newcommand{\linia}{\rule{\linewidth}{0.5pt}}

% custom theorems if needed
\newtheoremstyle{mytheor}
    {1ex}{1ex}{\normalfont}{0pt}{\scshape}{.}{1ex}
    {{\thmname{#1 }}{\thmnumber{#2}}{\thmnote{ (#3)}}}

\theoremstyle{mytheor}
\newtheorem{defi}{Definition}

% my own titles
\makeatletter

\renewcommand{\maketitle}{
\colorbox{gray!20}{\framebox[\linewidth]{ \huge \textsc{\@title} } 
\lfoot{\@title}
}

}
\makeatother
%%%



% custom footers and headers
\usepackage{fancyhdr}
\pagestyle{fancy}
\lhead{}
\chead{Métodos Numéricos - 1C2020}
\rhead{}
\cfoot{}
\rfoot{Pagina \thepage}
\renewcommand{\footrulewidth}{0.2pt}
%


\renewcommand\headrulewidth{.5pt}
\makeatletter
\def\headrule{{\if@fancyplain\let\headrulewidth\plainheadrulewidth\fi
\hrule\@height\headrulewidth\@width\headwidth
\vskip 1pt% 2pt between lines
\hrule\@height1.5pt\@width\headwidth% lower line with .5pt line width
\vskip-\headrulewidth
\vskip-1.5pt}}
\makeatother


\makeatletter

\newcommand{\inlineitem}[1][]{%
\ifnum\enit@type=\tw@
    {\descriptionlabel{#1}}
  \hspace{\labelsep}
\else
  \ifnum\enit@type=\z@
       \refstepcounter{\@listctr}\fi
    \quad\@itemlabel\hspace{\labelsep}
\fi}
\makeatother




% code listing settings
\usepackage{listings}
\lstset{
    language=Python,
    basicstyle=\ttfamily\small,
    aboveskip={1.0\baselineskip},
    belowskip={1.0\baselineskip},
    columns=fixed,
    extendedchars=true,
    breaklines=true,
    tabsize=4,
    prebreak=\raisebox{0ex}[0ex][0ex]{\ensuremath{\hookleftarrow}},
    frame=lines,
    showtabs=false,
    showspaces=false,
    showstringspaces=false,
    keywordstyle=\color[rgb]{0.627,0.126,0.941},
    commentstyle=\color[rgb]{0.133,0.545,0.133},
    stringstyle=\color[rgb]{01,0,0},
    numbers=left,
    numberstyle=\small,
    stepnumber=1,
    numbersep=10pt,
    captionpos=t,
    escapeinside={\%*}{*)}
}
 
%%%----------%%%----------%%%----------%%%----------%%%

\begin{document}


\title{Guía 6}

\author{Ulises Bussi-Javier Portillo, Universidad Nacional de Quilmes}


\maketitle \vspace{20pt}

\section*{Derivadas e Integrales}


\subsection*{Algoritmos}

\begin{enumerate}[label=\Roman*]

\item Programe una función que le permita calcular la derivada de primer orden hacia adelante o hacia atrás, tome como entrada un vector $x$, $y$ y una variable que le permita elegir si quiere la derivada hacia adelante o hacia atrás y devuelva el nuevo $x1$ y el nuevo $y'$ (en general son vectores de largo $\text{length(x1)}==\text{length(x1)}-1$).

\item Programe una función que le permita calcular la derivada segunda hacia adelante de un conjunto de datos.

\item Programe una función que le permita calcular la derivada primera centrada.


\item Cree una función que permita calcular la integral de una función $f(x)$ usando los metodos de simpson 1/3 y 3/8.

\end{enumerate}


\subsection*{Practica}

\begin{enumerate}
    \item Los datos entregados por la función "E1P5 " corresponden a la posición de un móvil en un plano en función del tiempo. halle y grafique:
    \begin{itemize}
    	\item La velocidad en función del tiempo en $x$ y en $y$.
    	\item La aceleración en función del tiempo en $x$ y en $y$.
    	\item Grafique la Trayectoria ($x$ vs. $y$), el plano de velocidades ($x'$ vs. $y'$) y el 					plano de aceleraciones($x''$ vs. $y''$).
    \end{itemize}
    
    
    \item Los datos entregados por la función "E2P5 " a un funcional que se quiere minimizar, para encontrar el mínimo una propuesta es buscar las raices de la derivada de dicho funcional. Calcule la derivada Numéricamente, y halle su raíz para hallar el mínimo del funcional. Compruebe que se trata del mínimo utilizando funciones propias del software.


	\item Integre la función g(x) entre a y b: $g(x)=\dfrac{f(b)-f(a)}{b-a}(x-a)$. ¿Que representa g(x)? ¿Que relación tiene con f(x)(tome una f(x) genérica)? ¿Que significado tiene el resultado de la integral?

	\item Evalue la siguiente integral:

\begin{equation*}
    \int_{0}^{\pi}sin(3x)dx
\end{equation*}

\begin{enumerate}[label=\alph*)]
    \item Analíticamente
    \item Usando los métodos: del trapecio, punto medio. Simpson 1/3 y Simpson 3/8 con item n=[1,2,3,4,5,6,7,8,9,10]
    \item Calcule el error relativo para cada metodo y cada n. Grafique.
    \item Analice y explique los valores obtenidos en el punto c)
\end{enumerate} 




\item La fuerza en el mástil de un velero puede ser representada por:

\begin{equation*}
f(z)=200\left(\dfrac{z}{5+z}\right)e^{\tfrac{-2z}{H}}   
\end{equation*}


\noindent donde $z$ es la elevación respecto a la cubierta y $H$ es la altura del mastil. La fuerza total ejercida sobre el mastil, puede calcularse integrando la función $f(z)$ a lo largo del mastil, es decir:

\begin{equation*}
    F=\int_{0}^{H}f(z)dz
\end{equation*}

\begin{enumerate}[label=\alph*)]
    \item Calcule la fuerza sobre un mastil de 30 metros utilizando el metodo del trapecio y simpson 1/3. 
    \item Calcule la diferencia porcentual entre ambos resultados para n=10
    \item Gráfique la diferencia calculada en (b) versus el número de pasos, variando el numero de pasos entre 5 y 100.
\end{enumerate} 

\end{enumerate}




\end{document}


