

\documentclass[a4paper,11pt]{article}

\usepackage[T1]{fontenc}
\usepackage[utf8]{inputenc}
\usepackage{graphicx}
\usepackage{xcolor}
\usepackage{tcolorbox}


\renewcommand\familydefault{\sfdefault}
\usepackage{tgheros}
\usepackage[defaultmono]{droidmono}

\usepackage{amsmath,amssymb,amsthm,textcomp}
%\usepackage{enumerate}
\usepackage{enumitem}
\usepackage{multicol}
\usepackage{tikz}

\usepackage{geometry}
\geometry{left=25mm,right=25mm,%
bindingoffset=0mm, top=20mm,bottom=20mm}


\linespread{1.3}

\newcommand{\linia}{\rule{\linewidth}{0.5pt}}

% custom theorems if needed
\newtheoremstyle{mytheor}
    {1ex}{1ex}{\normalfont}{0pt}{\scshape}{.}{1ex}
    {{\thmname{#1 }}{\thmnumber{#2}}{\thmnote{ (#3)}}}

\theoremstyle{mytheor}
\newtheorem{defi}{Definition}

% my own titles
\makeatletter

\renewcommand{\maketitle}{
\colorbox{gray!20}{\framebox[\linewidth]{ \huge \textsc{\@title} } 
\lfoot{\@title}
}

}
\makeatother
%%%

% custom footers and headers
\usepackage{fancyhdr}
\pagestyle{fancy}
\lhead{}
\chead{Métodos Numéricos - 1C2020}
\rhead{}
\cfoot{}
\rfoot{Pagina \thepage}
\renewcommand{\footrulewidth}{0.2pt}
%


\renewcommand\headrulewidth{.5pt}
\makeatletter
\def\headrule{{\if@fancyplain\let\headrulewidth\plainheadrulewidth\fi
\hrule\@height\headrulewidth\@width\headwidth
\vskip 1pt% 2pt between lines
\hrule\@height1.5pt\@width\headwidth% lower line with .5pt line width
\vskip-\headrulewidth
\vskip-1.5pt}}
\makeatother


\makeatletter

\newcommand{\inlineitem}[1][]{%
\ifnum\enit@type=\tw@
    {\descriptionlabel{#1}}
  \hspace{\labelsep}
\else
  \ifnum\enit@type=\z@
       \refstepcounter{\@listctr}\fi
    \quad\@itemlabel\hspace{\labelsep}
\fi}
\makeatother




% code listing settings
\usepackage{listings}
\lstset{
    language=Python,
    basicstyle=\ttfamily\small,
    aboveskip={1.0\baselineskip},
    belowskip={1.0\baselineskip},
    columns=fixed,
    extendedchars=true,
    breaklines=true,
    tabsize=4,
    prebreak=\raisebox{0ex}[0ex][0ex]{\ensuremath{\hookleftarrow}},
    frame=lines,
    showtabs=false,
    showspaces=false,
    showstringspaces=false,
    keywordstyle=\color[rgb]{0.627,0.126,0.941},
    commentstyle=\color[rgb]{0.133,0.545,0.133},
    stringstyle=\color[rgb]{01,0,0},
    numbers=left,
    numberstyle=\small,
    stepnumber=1,
    numbersep=10pt,
    captionpos=t,
    escapeinside={\%*}{*)}
}

%%%----------%%%----------%%%----------%%%----------%%%

\begin{document}

\title{Guía 0}

\author{Ulises Bussi, Universidad Nacional de Quilmes}

\date{01/01/2014}

\maketitle \vspace{20pt}


\textbf{Nota: } En esta guía se introducirán, ejercicios para el manejo básico del software, no necesariamente ejercicios de métodos numéricos estrictamente hablando.\\

\textbf{Nota 2: } Al realizar mediciones de tiempo en los algoritmos, muchas veces se verá la presencia de inestabilidades. Esto se debe a que el sistema operativo designa las prioridades de procesos, eso puede hacer que realizar la misma cuenta varias veces y medir el tiempo tome distintos tiempos (puede hacer el experimento si desea). Para reducir ese error, una solución simple es realizar la cuenta repetidas veces y promediar esos tiempos. Existen soluciones más complejas pero escapan de los alcances de la materia. 

\subsubsection*{\underline{Ejercicio 1}}

\begin{enumerate}[label=\alph*.]
 \item Cargue un vector con los valores $0$ al $100$ con paso $0.1$.
 \item Evalúe y grafique las siguientes funciones:
 \begin{enumerate}[label=\roman*.]
   \item $f_1(x) =x^2 +3x -7$  \hspace{115pt} \inlineitem $f_2(x) = e^x - 3x^2 \ln(x)$
   \item $f_3(x) = \sqrt{x^2+x+\cos(x)}\ e^{-x^2}$ \hspace{65pt} \inlineitem $f_4(x) = \sqrt{x^2 -3}$
   \item $f_5(x) = \cos x$
 \end{enumerate}
\end{enumerate}
Respecto a la $f_4$ que sucedió?.\\
Respecto a la $f_5$ que observación se puede hacer?  calcule el seno de $90^\circ$

\subsubsection*{\underline{Ejercicio 2}}
Se define el factorial de un número natural como:

\begin{equation*}
	\Gamma(n) = \left\lbrace
	\begin{array}{l c}
		1 								& \text{si } \text n =1 \vee  \text n =0\\
		n .\ \Gamma(n-1) 	& \text{si } \text n > 1 
	\end{array} \right.
\end{equation*}
Realice un programa que le permita calcular el factorial de un número.

\subsubsection*{\underline{Ejercicio 3}}

En el 1200 un viajero, comerciante y matemático planteó un problema comercial, cuanto menos interesante:

Supongamos que uno quiere dedicarse a la cría de conejos para la venta. Entonces inicialmente compra una pareja de conejos recién nacidos, unos conejos emhh... mágicos, estos conejos tienen algunas particularidades:
\begin{itemize}
\item cada conejo tarda, desde su nacimiento, un mes en llegar a la madurez sexual.
\item una vez en la madurez, estos conejos se reproducen, teniendo en cada apareamiento un par de crias.
\end{itemize}
Ignoremos el problema Adán y Eva de la reproducción entre hermanos y veamos como aumenta nuestra población de conejos. Al primer mes, tendremos la pareja de recién nacidos que compramos. 

Al segundo mes, esta pareja alcanza la madurez y se reproduce, pero el periodo de gestación es de un mes, con lo que seguimos teniendo una sola pareja. 

En el tercer mes tendremos 2 parejas :la inicial, procreando de nuevo, y una recién nacida. 

Para el cuarto mes tendremos 3 parejas: la inicial, la primer generación ya adulta y procreando y la segunda generación recién nacida.

En el quinto mes tendremos 5 parejas: la que compramos y la primer generación tuvieron crias, la segunda generación comienza a procrear.

Si siguen avanzando los meses podemos ver que el número de parejas será: 1,1,2,3,5,8,13... (básicamente el resultado de cada mes es la suma de los dos meses anteriores). A esta sucesión de números se la conoce como la sucesión de Fibonacci. Tiene su origen en el problema recién contado, por el comerciante viajero y matemático Leonardo de Pisa (más conocido por el nombre de Fibonacci).

Realice un programa que le devuelva los valores de la sucesión de Fibonacci hasta un orden n.\\

\begin{itemize}
\item[-] Calcule los primeros 50 términos de la sucesión, grafiquelos. Grafique también el cociente de cada término respecto del anterior ($ f_k / f_{k-1}$), incluya en este gráfico también una recta en el número $\varphi = 1.61803398875$
\end{itemize}


\subsubsection*{\underline{Ejercicio 4}}

Realice un programa que determine si un número a es divisible por otro número b, con $\text a, \text b \in \mathbb{N}$.

\subsubsection*{\underline{Ejercicio 5}}


\begin{enumerate}
\item Si queremos encontrar todos los divisores enteros de un número $n$ basta con probar si la división de $n$ con cada uno de los números hasta $n-1$ da un número entero (o tiene resto 0). 
\item Pero este proceso se puede optimizar, en realidad uno podría comparar contra todos los números enteros que sean menor que $l = n/2$ puesto que si un número  mayor que $l$ es divisor de $n$ el resultado sería un número $<2$ lo que no tiene sentido . Por lo tanto se puede reducir el numero de casos a probar (A la mitad!!).
\item Pero este proceso se puede optimizar (Si, aún más) , en realidad uno podría comparar contra todos los números enteros que sean menor que $k = \sqrt{n}$ puesto que si un número  mayor que $k$ es divisor de $n$ el resultado será un número $<k$. Por lo tanto se puede reducir el numero de casos a probar.


\end{enumerate}






 miremos un ejemplo paso a paso:

Supongamos que trabajamos con el número $46$ y queremos hallar todos sus divisores. podríamos probar con una calculadora y vamos a encontrar que las soluciones son: $[1, 2, 23, 46]$ el número de soluciones es 4! si hubiésemos probado con la primer forma, eran 44 intentos (el 1 y el 46 eran triviales pero los números del 2 al 45 no). Con el segundo método reducíamos a la mitad, 22 intentos (del 2 al 23). y por último, con el método del final eran unos $\text{floor} (\sqrt{46}) = 6$ intentos (en realidad 5, el 1 ya sabemos que cumple... ).  Esto no parece significativo en un principio, pero a medida que el número a evaluar se hace más y más grande, más se marca también la diferencia.


\begin{itemize}
\item Programar el método descrito en 1.
\item Programar el método descrito en 2.
\item Programar el método descrito en 3.
\item Realizar una medición de tiempo de cada uno de los tres métodos, y compararlos, para los siguientes números: $[9 , 53, 126,534,5313120]$

\end{itemize}




\subsubsection*{\underline{Ejercicio 6}}

Cuando se realiza el producto de dos matrices cuadradas $A$ y $B$, se calculan los coeficientes del resultado $C=A . B$ como:

$$c_{i,j} = \sum_{k=0}^n a_{i,k}\  b_{k,j}$$

\begin{itemize}
\item Realice un programa para calcular el producto de dos matrices cuadradas. 
\item Mida el tiempo que tarda dicho programa en realizar el producto con matrices de orden: $[5,10,20,50,75,100,150,200,250,300,400,500]$.
\item Compare utilizando el producto propio del software (en MatLab es $A*B$).
\end{itemize}







\end{document}