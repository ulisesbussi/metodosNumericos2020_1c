\documentclass[a4paper,11pt]{article}

\usepackage[T1]{fontenc}
\usepackage[utf8]{inputenc}
\usepackage{graphicx}
\usepackage{xcolor}
\usepackage{tcolorbox}
\usepackage{sectsty}
\usepackage{enumitem}
\usepackage{circuitikz}
\renewcommand\familydefault{\sfdefault}
\usepackage{tgheros}
\usepackage{amsmath,amssymb,amsthm,textcomp}
\usepackage{enumitem}
\usepackage{multicol}
\usepackage{tikz}

\usepackage{geometry}
\geometry{left=25mm,right=25mm,%
bindingoffset=0mm, top=20mm,bottom=20mm}





\linespread{1.3}

\newcommand{\linia}{\rule{\linewidth}{0.5pt}}

% custom theorems if needed
\newtheoremstyle{mytheor}
    {1ex}{1ex}{\normalfont}{0pt}{\scshape}{.}{1ex}
    {{\thmname{#1 }}{\thmnumber{#2}}{\thmnote{ (#3)}}}

\theoremstyle{mytheor}
\newtheorem{defi}{Definition}

% my own titles
\makeatletter

\renewcommand{\maketitle}{
\colorbox{gray!20}{\framebox[\linewidth]{ \huge \textsc{\@title} } 
\lfoot{\@title}
}

}
\makeatother
%%%



% custom footers and headers
\usepackage{fancyhdr}
\pagestyle{fancy}
\lhead{}
\chead{Métodos Numéricos - 1C2020}
\rhead{}
\cfoot{}
\rfoot{Pagina \thepage}
\renewcommand{\footrulewidth}{0.2pt}
%


\renewcommand\headrulewidth{.5pt}
\makeatletter
\def\headrule{{\if@fancyplain\let\headrulewidth\plainheadrulewidth\fi
\hrule\@height\headrulewidth\@width\headwidth
\vskip 1pt% 2pt between lines
\hrule\@height1.5pt\@width\headwidth% lower line with .5pt line width
\vskip-\headrulewidth
\vskip-1.5pt}}
\makeatother


\makeatletter

\newcommand{\inlineitem}[1][]{%
\ifnum\enit@type=\tw@
    {\descriptionlabel{#1}}
  \hspace{\labelsep}
\else
  \ifnum\enit@type=\z@
       \refstepcounter{\@listctr}\fi
    \quad\@itemlabel\hspace{\labelsep}
\fi}
\makeatother




% code listing settings
\usepackage{listings}
\lstset{
    language=Python,
    basicstyle=\ttfamily\small,
    aboveskip={1.0\baselineskip},
    belowskip={1.0\baselineskip},
    columns=fixed,
    extendedchars=true,
    breaklines=true,
    tabsize=4,
    prebreak=\raisebox{0ex}[0ex][0ex]{\ensuremath{\hookleftarrow}},
    frame=lines,
    showtabs=false,
    showspaces=false,
    showstringspaces=false,
    keywordstyle=\color[rgb]{0.627,0.126,0.941},
    commentstyle=\color[rgb]{0.133,0.545,0.133},
    stringstyle=\color[rgb]{01,0,0},
    numbers=left,
    numberstyle=\small,
    stepnumber=1,
    numbersep=10pt,
    captionpos=t,
    escapeinside={\%*}{*)}
}

%%%----------%%%----------%%%----------%%%----------%%%

\begin{document}


\title{Guía 3}

\author{Ulises Bussi-Javier Portillo, Universidad Nacional de Quilmes}


\maketitle \vspace{20pt}

\section*{Sistemas De Ecuaciones: métodos directos}
%

\subsection*{Algoritmos}

\begin{enumerate}[label=\Roman*]

\item \label{enum:remonte} Aplique el algoritmo de remonte explicado en clase para encontrar la solución del sistema de ecuaciones de forma $U x =b$ donde U es una matriz triangular superior. Esta función debe tomar como entrada una matriz $U$ y el vector $b$ y devolver el vector $x$.

\item \label{enum:descenso} Aplique el algoritmo de descenso para encontrar la solución del sistema de ecuaciones de forma $L x =b$ donde U es una matriz triangular inferior.Esta función debe tomar como entrada una matriz $L$ y el vector $b$ y devolver el vector $x$.



\item \label{enum:gaussJordan} Cree una función que permita resolver el problema $A x=b$ utilizando Gauss-Jordan.

\item \label{enum:pivot} Cree una función que permita resolver el problema $A x=b$ utilizando Gauss-Jordan con pivoteo parcial.


\item \label{enum:LU} Cree una función que tome como entrada una matriz $A$ y devuelva las matrices $L$ y $U$ propias del algoritmo de descomposición LU.


\item \label{enum:errStop} Cree una nueva función que llame a \ref{enum:remonte},\ref{enum:descenso} y \ref{enum:LU} para resolver el problema $A x=b$


\end{enumerate}

\subsection*{Ejercicio 1}
Determine el numero de operaciones de punto flotante para los siguientes algoritmos para matrices de $n\times	n$:
\begin{itemize}
\item Algoritmo de Remonte.
\item Algoritmo de Descenso.
\item Factorización LU.
\end{itemize}

\subsection*{Ejercicio 2}
Encuentre la solución a los siguentes sistemas de ecuaciones aplicando Gauss-Jordan:

\begin{itemize}
\item[a] $\left\lbrace \begin{array}{ccc}
10x_1 +2x_2 -x_3 & = & 27 \\
-3x_1 -6x_2 +2x_3 & = & -61.5 \\
x_1 +x_2 +5x_3 & = & -21.5 \\
\end{array} \right.$ 
\item[b] $\left\lbrace \begin{array}{ccc}
-2x_2 +x_3 & = & -10 \\
2x_1 +6x_2 -4x_3 & = & 44 \\
-x_1 -2x_2 +5x_3 & = & -26 \\
\end{array} \right.$
\end{itemize}


\subsection*{Ejercicio 3}

Resuelva el ejercicio 2a, pero utilizando descompusición LU, para los siguientes vectores solucion (b):

$b_1= \begin{bmatrix}
12\\
18\\
-6
\end{bmatrix},
b_2= \begin{bmatrix}
3\\
-4\\
-7
\end{bmatrix},
b_3= \begin{bmatrix}
1\\
0\\
-1
\end{bmatrix}$



\section*{Sistemas De Ecuaciones: métodos iterativos}

\subsection*{Algoritmos}

\begin{enumerate}[label=\Roman*]

\item Cree una función que analice si el algoritmo iterativo de Gauss-Seidel convergerá.

\item Cree una función que realice pivoteo de filas (solo con las filas que siguen) si no cumple con la condición de convergencia.

\item Cree una función que realice el algoritmo de Gauss-Seidel clásico, checkeando previamente la convergencia (recuerde que es condicion suficiente pero no necesaria, así que solo muestre un mensaje por pantalla avisando esto). la Función debe tomar como parmaetro de entrada la matriz $A$ el vector $B$, un vector de tolerancia de errores y un número máximo de iteraciones para salir en caso de que no converja. 

\item Cree una función que realice el algoritmo de Gauss-Seidel con relajación, permitiendo elegir el parámetro de relajación.
 
\end{enumerate}


\subsection*{Ejercicio 4}

Determine el numero de operaciones de punto flotante para los siguientes algoritmos para matrices de $n\times	n$:
\begin{itemize}
\item Gauss-Seidel.
\item Gauss-Seidel con Relajación.
\end{itemize}

\subsection*{Ejercicio 5}


Resuelva utilizando Gauss-Seidel el siguiente sistema de ecuaciones 
$$\begin{array}{ccc}
10x_1 +2x_2 - x_3 	&=& 27\\
-3x_1 -6x_2 + 2x_3 	&=& -61.5\\
x_1+x_2+ 5x_3 		&=& -21.5
\end{array}$$

\subsection*{Ejercicio 6}

Resuelva el siguiente sistema de ecuaciones:

$$\begin{array}{ccc}
2x_1 -6x_2 -x_3 &=& -38\\
-3x_1-x_2+7x_3 &=& -34\\
-8x_1 +x_2-2x_3 &=& -20
\end{array}$$

Justifique que método utilizó para resolverlo.


\end{document}
