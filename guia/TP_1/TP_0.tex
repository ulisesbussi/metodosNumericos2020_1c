\documentclass[a4paper, 11pt]{article}
\usepackage{comment} % enables the use of multi-line comments (\ifx \fi) 
\usepackage{fullpage} % changes the margin
\usepackage[spanish]{babel}
\selectlanguage{spanish}
\usepackage[utf8]{inputenc}
\usepackage{graphicx}
\usepackage{enumitem}
\usepackage{amsmath}
 

\begin{document}
\renewcommand{\listtablename}{Índice de tablas} 
\renewcommand{\tablename}{Tabla}
\begin{center}
\LARGE \bf Trabajo práctico número 1: \\ Vectores, matrices y funciones
\end{center}

\vspace{1cm} 
\noindent\textbf{Materia:} M\'etodos Num\'ericos \hfill
\textbf{Año} 2020 - 1C \\
\hfill \\
%\textbf{Fecha} \today \hfill \textbf{Ayudante:} Diego Hidalgo  \\ 
%\vspace{1cm} 
\section*{Vectores}

\begin{enumerate}[font=\large\bfseries]
    \item \\
    \begin{enumerate}
    	\item Crear el vector [7 1  -2 2 4 9 12 -2 8 0 -1 -5 6 0 9 -3 -5 10 -4 8] y guardarlo en la variable x
    	\item calcular el numero de elemetos del vector y guargarlo en la variable n
    	\item sumar los elementos 2, 4 y 6 del vector x
    	\item sumar los elementos 3,6,...,15, 18 del vector (utilizar un for)
    	\item sumar los elementos 5, 9, 13, 17 (utilizando un for)
    	\item crear una funcion que encuentre el valor máximo de x
    	\item crear una funcion que encuentre la mediana del vector y el rango (valor máximo y mínimo)
	\end{enumerate}
	\item
    \begin{enumerate}
        \item Crear un vector de la forma [1 2 3 4 ... 100]
        \item Crear un vector con 100 elementos, de la forma [1, -1, 3, -3, 5, -5 ... 49, -49]
        \item Crear un vector con 100 elementos, de la forma [1, -2, 4, -16, ..., -1048576]
        \item Crear un vector con 100 elementos, de la forma [1, 100, 2, 99, ..., 50, 51]
    \end{enumerate}
    
\section*{Matrices }

    \item
    \begin{enumerate}
        \item Crear la siguiente matriz:
        
                \[
                M=
                  \begin{bmatrix}
                    1 & 4 & 2 & 5 \\
                    8 & 12 & 7 & 8
                  \end{bmatrix}
                \]
    
    
    \item ¿Cuál es el tamaño de la matriz?
    \item Cambiar los valores necesarios para que todos los elementos de la matriz sean pares
    \end{enumerate}
    \item
    \begin{enumerate}
        \item Crear las siguientes matrices (utilizando en ambos casos 2 "for"):
        
                \[
                M1=
                  \begin{bmatrix}
                    1 & 2 & 3 & 4 & 5 \\
                    2 & 1 & 2 & 3 & 4 \\
                    3 & 2 & 1 & 2 & 3 \\
                    4 & 3 & 2 & 1 & 2 \\
                    5 & 4 & 3 & 2 & 1
                  \end{bmatrix}
                 %
                \hspace{2cm} M2=\begin{bmatrix}
                    1 & 4 & 9 & 16 & 25 \\
                    4 & 1 & 4 & 9 & 16 \\
                    9 & 4 & 1 & 4 & 9 \\
                    16 & 9 & 4 & 1 & 4 \\
                    25 & 16 & 9 & 4 & 1
                  \end{bmatrix}
                \]
    \end{enumerate}
    
    
\section*{Funciones}

\item Crear una función que, dado un numero, calcule su factorial
\item Crear una función que determine si un número es primo
\item Crear una función que tome los valores n y m y cree una matriz A, tal que A(m,n)=m-2*n

\item
\begin{enumerate}
    \item Crear una función que tome la matriz A y devuelva la matriz A*, tal que:
    \[
      A=\begin{bmatrix}
        A_{1,1} & A_{1,2} & A_{1,3} & A_{1,4} \\
        A_{2,1} & A_{2,2} & A_{2,3} & A_{2,4} \\
        A_{3,1} & A_{3,2} & A_{3,3} & A_{3,4} \\
        A_{4,1} & A_{4,2} & A_{4,3} & A_{4,4}
     \end{bmatrix}
     %
    \hspace{1cm} \LARGE\rightarrow\normalsize
    \hspace{1cm} A*=\begin{bmatrix}
        A_{1,1} & 0 & 0 & 0 \\
        A_{1,2} & A_{2,2} & 0 & 0 \\
        A_{1,3} & A_{2,3} & A_{3,3} & 0 \\
        A_{1,4} & A_{2,4} & A_{3,4} & A_{4,4}
     \end{bmatrix}
     \]\\
      
    \item Generalizar la funcion anterior para matrices de NxN 
\end{enumerate}
    
\end{enumerate}



\end{document}

